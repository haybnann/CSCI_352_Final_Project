%% bare_jrnl_compsoc.tex
%% V1.4a
%% 2014/09/17
%% by Michael Shell
%% See:
%% http://www.michaelshell.org/
%% for current contact information.
%%
%% This is a skeleton file demonstrating the use of IEEEtran.cls
%% (requires IEEEtran.cls version 1.8a or later) with an IEEE
%% Computer Society journal paper.
%%
%% Support sites:
%% http://www.michaelshell.org/tex/ieeetran/
%% http://www.ctan.org/tex-archive/macros/latex/contrib/IEEEtran/
%% and
%% http://www.ieee.org/

%%*************************************************************************
%% Legal Notice:
%% This code is offered as-is without any warranty either expressed or
%% implied; without even the implied warranty of MERCHANTABILITY or
%% FITNESS FOR A PARTICULAR PURPOSE! 
%% User assumes all risk.
%% In no event shall IEEE or any contributor to this code be liable for
%% any damages or losses, including, but not limited to, incidental,
%% consequential, or any other damages, resulting from the use or misuse
%% of any information contained here.
%%
%% All comments are the opinions of their respective authors and are not
%% necessarily endorsed by the IEEE.
%%
%% This work is distributed under the LaTeX Project Public License (LPPL)
%% ( http://www.latex-project.org/ ) version 1.3, and may be freely used,
%% distributed and modified. A copy of the LPPL, version 1.3, is included
%% in the base LaTeX documentation of all distributions of LaTeX released
%% 2003/12/01 or later.
%% Retain all contribution notices and credits.
%% ** Modified files should be clearly indicated as such, including  **
%% ** renaming them and changing author support contact information. **
%%
%% File list of work: IEEEtran.cls, IEEEtran_HOWTO.pdf, bare_adv.tex,
%%                    bare_conf.tex, bare_jrnl.tex, bare_conf_compsoc.tex,
%%                    bare_jrnl_compsoc.tex, bare_jrnl_transmag.tex
%%*************************************************************************


% *** Authors should verify (and, if needed, correct) their LaTeX system  ***
% *** with the testflow diagnostic prior to trusting their LaTeX platform ***
% *** with production work. IEEE's font choices and paper sizes can       ***
% *** trigger bugs that do not appear when using other class files.       ***                          ***
% The testflow support page is at:
% http://www.michaelshell.org/tex/testflow/


\documentclass[10pt,conference,onecolumn,compsoc]{IEEEtran}


\usepackage{hyperref}
\usepackage{enumitem}
\setlist[itemize]{leftmargin=3 cm}
\setlist[enumerate]{leftmargin=3cm}



% *** CITATION PACKAGES ***
%
\ifCLASSOPTIONcompsoc
  % IEEE Computer Society needs nocompress option
  % requires cite.sty v4.0 or later (November 2003)
  \usepackage[nocompress]{cite}
\else
  % normal IEEE
  \usepackage{cite}
\fi
% cite.sty was written by Donald Arseneau
% V1.6 and later of IEEEtran pre-defines the format of the cite.sty package
% \cite{} output to follow that of IEEE. Loading the cite package will
% result in citation numbers being automatically sorted and properly
% "compressed/ranged". e.g., [1], [9], [2], [7], [5], [6] without using
% cite.sty will become [1], [2], [5]--[7], [9] using cite.sty. cite.sty's
% \cite will automatically add leading space, if needed. Use cite.sty's
% noadjust option (cite.sty V3.8 and later) if you want to turn this off
% such as if a citation ever needs to be enclosed in parenthesis.
% cite.sty is already installed on most LaTeX systems. Be sure and use
% version 5.0 (2009-03-20) and later if using hyperref.sty.
% The latest version can be obtained at:
% http://www.ctan.org/tex-archive/macros/latex/contrib/cite/
% The documentation is contained in the cite.sty file itself.



% *** GRAPHICS RELATED PACKAGES ***
%
\ifCLASSINFOpdf
   \usepackage[pdftex]{graphicx}
 
\else
 
\fi
% graphicx was written by David Carlisle and Sebastian Rahtz. It is
% required if you want graphics, photos, etc. graphicx.sty is already
% installed on most LaTeX systems. The latest version and documentation
% can be obtained at: 
% http://www.ctan.org/tex-archive/macros/latex/required/graphics/
% Another good source of documentation is "Using Imported Graphics in
% LaTeX2e" by Keith Reckdahl which can be found at:
% http://www.ctan.org/tex-archive/info/epslatex/
%
% latex, and pdflatex in dvi mode, support graphics in encapsulated
% postscript (.eps) format. pdflatex in pdf mode supports graphics
% in .pdf, .jpeg, .png and .mps (metapost) formats. Users should ensure
% that all non-photo figures use a vector format (.eps, .pdf, .mps) and
% not a bitmapped formats (.jpeg, .png). IEEE frowns on bitmapped formats
% which can result in "jaggedy"/blurry rendering of lines and letters as
% well as large increases in file sizes.
%
% You can find documentation about the pdfTeX application at:
% http://www.tug.org/applications/pdftex









% *** PDF, URL AND HYPERLINK PACKAGES ***
%
\usepackage{url}
% url.sty was written by Donald Arseneau. It provides better support for
% handling and breaking URLs. url.sty is already installed on most LaTeX
% systems. The latest version and documentation can be obtained at:
% http://www.ctan.org/tex-archive/macros/latex/contrib/url/
% Basically, \url{my_url_here}.




\begin{document}

\title{Project Edward: Final Project Documentation \\ for UTM CSCI 352}
%
%

% received ..."  text while in non-compsoc journals this is reversed. Sigh.

\author{Hayden Nanney \& Matt Mosley% <-this % stops a space
}

\IEEEtitleabstractindextext{%
\begin{abstract}
This project is an application for those who are planning for their retirement at any point in their career. The goal of the prject is to allow the user's to set retirement saving's goals and track their progress along the way. It can also be used to forcast what your savings will be if you adhere to a certain investment strategy. 
\end{abstract}

}


% make the title area
\maketitle



\IEEEdisplaynontitleabstractindextext

\IEEEpeerreviewmaketitle



\section{Introduction}
This project will allow a user to enter in financial information such as income, filing status, year, as well as whether they are contributing to a pretax retirement savings account to accurately calculated taxes and tax deductions for each tax year. This will allow the app to accurately reflect its user's financial history as well as help them to plan for their retirement while early in their career, which is most beneficial due to compounding interest.

\subsection{Background}
This application may include large amounts of financial jargon such as marginal tax rates, adjusted gross income, etc. We aim to make definitions and explanations of these terms available in app, so we can make this information more accessible. 

\subsection{Impacts}
This application should impact the financial literacy of its users, since financial education is sorely lacking in most education programs.

\subsection{Challenges}
We expect some headaches when we start to implement the tax laws and regulations since it will be time consuming to get everything correct and easily changeable.

Current challenges are how we should document all of the tax brackets and we are debating on using a database with each table storing one year's tax bracket data.

\section{Scope}
At the completion of this project the end user will be able to enter information that will affect their tax status such as salary, filing status, and location and then the app will project how this will impact their retirement savings accounts and provide the user a resource to show how much they need to save and invest per year to meet their goals.
\subsection{Requirements}
The following requirements were generated based on the information needed to make financial advice.

\subsubsection{Functional}
\begin{itemize}
\item Each user will have a profile that contains a list of the user's yearly tax history.
\end{itemize}

\subsubsection{Non-Functional}
\begin{itemize}
\item The user will have an account number, password, and other private information.
\end{itemize}

\subsection{Use Cases}
\begin{table}[ht]
\centering
\begin{tabular}{|c|c|c|c|c|}
\hline
Use Case ID & Use Case Name & Primary Actor & Complexity & Priority \\
\hline \hline
1 & View Profile & User & Low & 1\\
\hline
2 & Salary & User & Low & 1\\
\hline

\end{tabular}
\caption{Use case table}
\label{tab:useCaseIndex}
\end{table}


\begin{itemize}
	\item[Use Case Number:] 1
	\item[Use Case Name:] View Profile
	\item[Description:] The user on our application wishes to view their account and the information contained in there. They will click on the "View Profile" button. Which will then refer them to their profile screen with their information contained in it.
\end{itemize}

\begin{enumerate}
	\item User wishes to see their account profile, first they would need to log into their account.
	\item The user then clicks on the View Profile button.
	\item The user is then directed to the profile page where they can view their information
	\item[Termination Outcome:] The user now is able to view their profile and view their information.
\end{enumerate} 


%item 2
\begin{itemize}
	\item[]  
	\item[Use Case Number:] 2
	\item[Use Case Name:] Salary
	\item[Description:] The user in the application's home screen can tap on this button to view their current salary and how much they are earning, along with their paychecks and incoming and outgoing payments.
\end{itemize}

\begin{enumerate}
	\item User wishes to see their salary.
	\item User clicks on the View Salary button.
	\item The user is directed to the Salary page where they can view their information
	\item[Termination Outcome:] The user now can view their salary information.
\end{enumerate} 

%item 3
\begin{itemize}
	\item[]
	\item[Use Case Number:] 3
	\item[Use Case Name:] Home
	\item[Description:] The user will be able to go to the home screen to see updates about their information, including salary and transactions.
\end{itemize}

\begin{enumerate}
	\item The user wishes to view the home page.
	\item When the user loads up the application, this page is default, but if they are in another tab, they will click on the Home button.
	\item The user is directed to the Home page of the application.
	\item[Termination Outcome:] The user can now view the home screen and see their information.
\end{enumerate}
\newpage
\subsection{Interface Mockups}
\begin{figure}[h]
	\centering
	\includegraphics[width=0.7\linewidth]{uihome}
	\caption{Login Page}
	\label{fig:uihome}
\end{figure}
\begin{figure}[h]
	\centering
	\includegraphics[width=0.7\linewidth]{uihomecreen}
	\caption{Home Page}
	\label{fig:uihomecreen}
\end{figure}
\begin{figure}[h]
	\centering
	\includegraphics[width=0.7\linewidth]{financialplan}
	\caption{Financial Plan Page}
	\label{fig:financialplan}
\end{figure}

\newpage
\section{Project Timeline}
We will add project timeline in the next update.

\section{Project Structure}
We will add the project structure in the next update.

\subsection{UML Outline}

\begin{figure}[h]
	\centering
	\includegraphics[width=0.7\linewidth]{backend_uml}
	\caption{Backend UML}
	\label{fig:backenduml}
\end{figure}
\begin{figure}[h]
	\centering
	\includegraphics[width=0.7\linewidth]{ui_uml}
	\caption{UI UML}
	\label{fig:uiuml}
\end{figure}
\newpage



\subsection{Design Patterns Used}
To do next submission.


\section{Results}
We have a general idea of where we want our project to head towards. We will be brainstorming more ideas and will document them as we complete each goal that we have set for the next upcoming deliverables.

\subsection{Future Work}
The next steps are to create a UML diagram of the project and get a minimum viable product working that allows the user to submit a salary amount and returns their predicted after-tax income.


\begin{IEEEbiography}{Michael Shell}
Biography text here.
\end{IEEEbiography}

% if you will not have a photo at all:
\begin{IEEEbiographynophoto}{John Doe}
Biography text here.
\end{IEEEbiographynophoto}

% insert where needed to balance the two columns on the last page with
% biographies
%\newpage

\begin{IEEEbiographynophoto}{Jane Doe}
Biography text here.
\end{IEEEbiographynophoto}





% that's all folks
\end{document}
